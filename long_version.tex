%% start of file `template.tex'.
%% Copyright 2006-2013 Xavier Danaux (xdanaux@gmail.com).
%
% This work may be distributed and/or modified under the
% conditions of the LaTeX Project Public License version 1.3c,
% available at http://www.latex-project.org/lppl/.


\documentclass[10pt,letterpaper,sans]{moderncv}        % possible options include font size ('10pt', '11pt' and '12pt'), paper size ('a4paper', 'letterpaper', 'a5paper', 'legalpaper', 'executivepaper' and 'landscape') and font family ('sans' and 'roman')

% moderncv themes
\moderncvstyle{classic}                            % style options are 'casual' (default), 'classic', 'oldstyle' and 'banking'
\moderncvcolor{blue}                               % color options 'blue' (default), 'orange', 'green', 'red', 'purple', 'grey' and 'black'
%\renewcommand{\familydefault}{\sfdefault}         % to set the default font; use '\sfdefault' for the default sans serif font, '\rmdefault' for the default roman one, or any tex font name
%\nopagenumbers{}                                  % uncomment to suppress automatic page numbering for CVs longer than one page

% character encoding
\usepackage[utf8]{inputenc}                       % if you are not using xelatex ou lualatex, replace by the encoding you are using
%\usepackage{CJKutf8}                              % if you need to use CJK to typeset your resume in Chinese, Japanese or Korean

% adjust the page margins
\usepackage[scale=0.85]{geometry}
\setlength{\hintscolumnwidth}{3cm}                % if you want to change the width of the column with the dates
%\setlength{\makecvtitlenamewidth}{10cm}           % for the 'classic' style, if you want to force the width allocated to your name and avoid line breaks. be careful though, the length is normally calculated to avoid any overlap with your personal info; use this at your own typographical risks...

% personal data
\name{Firas}{Abuzaid}
\title{Résumé}                                     % optional, remove / comment the line if not wanted
\address{P.O. Box 14167}{Stanford, CA 94305}% optional, remove / comment the line if not wanted; the "postcode city" and and "country" arguments can be omitted or provided empty
\phone[mobile]{+1~(214)~536~4585}                  % optional, remove / comment the line if not wanted
\phone[fixed]{+1~(775)~238~3228}                   % optional, remove / comment the line if not wanted
\email{fabuzaid@stanford.edu}                      % optional, remove / comment the line if not wanted
\homepage{github.com/fabuzaid21}                   % optional, remove / comment the line if not wanted
\photo[48pt][0.4pt]{firas}                  % optional, remove / comment the line if not wanted; '64pt' is the height the picture must be resized to, 0.4pt is the thickness of the frame around it (put it to 0pt for no frame) and 'picture' is the name of the picture file

% to show numerical labels in the bibliography (default is to show no labels); only useful if you make citations in your resume
%\makeatletter
%\renewcommand*{\bibliographyitemlabel}{\@biblabel{\arabic{enumiv}}}
%\makeatother
%\renewcommand*{\bibliographyitemlabel}{[\arabic{enumiv}]}% CONSIDER REPLACING THE ABOVE BY THIS

% bibliography with mutiple entries
%\usepackage{multibib}
%\newcites{book,misc}{{Books},{Others}}
%----------------------------------------------------------------------------------
%            content
%----------------------------------------------------------------------------------
\begin{document}
%\begin{CJK*}{UTF8}{gbsn}                          % to typeset your resume in Chinese using CJK
%-----       resume       ---------------------------------------------------------

\makecvtitle

\vspace{-12mm}

\section{Education}
\cventry{2013--present}{M.S., Computer Science}{Stanford University}{}{\textit{GPA: 3.936}}{Concentrations: Artificial Intelligence, Information}  % arguments 3 to 6 can be left empty
\cventry{2009--2013}{B.S., Computer Science}{Stanford University}{}{\textit{GPA: 3.776}}{Concentration: Information
\begin{itemize}
\item Phi Beta Kappa, Tau Beta Pi
\end{itemize}
}

\section{Work Experience}
\subsection{Research}
\cventry{04/2014--present}{Research Assistant}{Stanford InfoLab}{Prof. Christopher Ré}{}{
\begin{itemize}
\item Worked on multi-round Pregel-like join algorithm for distributed data sets
\item \textbf{Technologies and Libraries}: Scala, Spark, SparkSQL, Hive, YARN, Hadoop/HDFS
\end{itemize}
}
\cventry{01/2014--03/2014}{Independent Research Project}{Stanford AI Lab}{Prof. Andrew Ng}{}{
\begin{itemize}
\item Worked on re-alignment improvements for deep neural networks on speech recognition systems
\item \textbf{Technologies and Libraries}: C++, Python, Kaldi, Google Web Speech API
\end{itemize}
}
\subsection{Industry}
\cventry{06/2012--09/2012}{Android Engineer Intern}{Clinkle}{Mountain View, CA}{}{Implemented initial User Interface for the Clinkle Android application}
\cventry{06/2011--09/2011}{Software Engineer Intern}{Lab126}{Cupertino, CA}{}{Developed QuickSettings app for Kindle Fire; debugged, refined UI/UX for other system apps on Android 2.3.3}
\subsection{Teaching}
\cventry{2014--present}{Awards}{}{}{}{
\begin{itemize}
\item {Recipient of the \href{https://teachingcommons.stanford.edu/grants-awards/teaching-awards/centennial-teaching-assistant-awards}{\underline{2014 Centennial Teaching Assistant Award}}}
\end{itemize}
}
\cventry{2014--present}{Instructor}{Stanford Computer Science Department}{}{}{
\begin{itemize}
\item {\textbf{CS145}, \emph{Introduction to Databases}, Co-Instructor with Perth Charernwatttanagul, Summer 2014}
\end{itemize}
}
\cventry{2014--present}{Mentor in Teaching Fellow}{Stanford Computer Science Deparmtnet}{}{}{Responsible for assisting and mentoring new Teaching Assistants for the CS department}
\cventry{2013--present}{Teaching Assistant}{Stanford Computer Science Department}{}{}{
\begin{itemize}
\item {\textbf{CS142}, \emph{Web Applications}, Profs. John Ousterhout and Phillip Levis, Spring 2013 - Spring 2014}
\item {\textbf{CS145}, \emph{Introduction to Databases}, Profs. Christopher Ré and Jennifer Widom, Fall 2013 - Fall 2014}
\end{itemize}
}
\cventry{2011--2012}{CS198 Section Leader}{Stanford Computer Science Department}{}{}{Responsible for leading weekly discussion sections to complement lecture for introductory Computer Science classes -- CS106A, B, and X}
\subsection{Other}
\cventry{01/2014--present}{Master's Student Liaison}{Stanford Computer Science Department}{}{}{Responsible for communicating and voicing students' feedback and concerns to CS faculty}
\cventry{2011--2012}{Academic Theme Associate}{Stanford Residential Education}{}{}{Residential staff position -- responsible for creating and planning theme-related programming for the \href{http://studentaffairs.stanford.edu/resed/profiles/theme/crothers}{Crothers Global Citizenship dorm}}
\section{Individual Projects}
\cventry{09/2013}{NYTimes Headlines}{}{}{}{
\begin{itemize}
\item Self-developed Rails app
\item Lets users explore the popularity over time of different keywords in the New York Times' headlines, à la Google Ngram Viewer
\item \textbf{Technologies and Libraries}: Rails, D3.js, NVD3.js, Coffeescript, JQuery, CouchDB
\item Link: \href{http://nytimes-headlines.herokuapp.com/}{\underline{http://nytimes-headlines.herokuapp.com/}}
\item Browse source code here: \href{https://github.com/fabuzaid21/nytimes-headlines}{\underline{https://github.com/fabuzaid21/nytimes-headlines}}
\end{itemize}
}
\cventry{02/2013}{BuzzBuddy}{}{}{}{
\begin{itemize}
\item Self-developed Android app, over 2000 downloads
\item Lets users create custom vibration patterns for any of your favorite apps. When push notification is received, Buzz Buddy plays back the corresponding pattern -- indicating to user what type of notification (e-mail, text message, tweet, etc.) he/she has received
\item \textbf{Technologies and Libraries}: Android, ActionBarSherlock, StickyListHeaders, ACRA
\item Download here: \newline{}\href{https://play.google.com/store/apps/details?id=com.buzzbuddy.android}{\underline{https://play.google.com/store/apps/details?id=com.buzzbuddy.android}}
\item Browse source here: \href{https://github.com/fabuzaid21/BuzzBuddy}{\underline{https://github.com/fabuzaid21/BuzzBuddy}}
\end{itemize}
}

\section{Other Computer Skills}
\cvitemwithcomment{Mobile frameworks}{Windows Phone}{}
\cvitemwithcomment{Web frameworks}{Django}{}
\cvitemwithcomment{NoSQL}{Redis, HBase, MongoDB}{}
\cvitemwithcomment{Coursera courses}{Functional Programming Principles in Scala}{\href{https://www.dropbox.com/s/avspos66keqsq8o/FP_in_Scala_Grade.png/}{Earned 100.0\% with distinction}}

\section{Other Awards, Skills, and Interests}
\cvitem{Awards}{Stanford Human Rights Fellowship (2013), National Semifinalist in Intel Science Talent Search (2009), Regional Semifinalist in Siemens Research Competition (2008)}
\cvitem{Puzzles}{Rubik's Cube (fastest time: 45 seconds), Chess, 15-puzzle, Sudoku, Kakuro, Word Scramble}
\cvitem{Languages}{Arabic (fluent), French (semi-fluent)}
\cvitem{Music}{Clarinet player for 8 years, 2008 member of the 2008 Texas All-State Band}

\end{document}
%% end of file `template.tex'.
